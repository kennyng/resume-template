% !TEX TS-program = xelatex
% !TEX encoding = UTF-8 Unicode
% -*- coding: UTF-8; -*-
% vim: set fenc=utf-8

%%%%%%%%%%%%%%%%%%%%%%%%%%%%%%%%%%%%%%%%%%%%%%%%%%%%%%%%%%%%%%%%%%%%%%%%%%%%%%
%% resume.tex
%% -----------
%% LaTeX Resume (Curriculum Vitae)
%% > Compile with: `xelatex resume.tex`
%%
%% Author: Kenny Ng
%%%%%%%%%%%%%%%%%%%%%%%%%%%%%%%%%%%%%%%%%%%%%%%%%%%%%%%%%%%%%%%%%%%%%%%%%%%%%%


\documentclass[letterpaper,10pt]{article}

% ============================================================================
% LOAD PACKAGES
% ============================================================================
% For configuring page margins.
\usepackage[top=1.0cm,left=1.5cm,right=1.5cm,bottom=1.0cm]{geometry}
% For loading fonts.
\usepackage{fontspec}
% For custom sections.
\usepackage{titlesec}
% For handling hyperlinks.
\usepackage[hidelinks]{hyperref}
\hypersetup{pdftitle={Kenny Ng - Résumé}, pdfauthor={Kenny Ng}}
% For icon fonts.
\usepackage{fontawesome}
% For fixed length tables.
\usepackage{array}
% For handling text alignment.
\usepackage{ragged2e}
% For dealing with paragraphs.
\usepackage{parskip}
% For handling list environment.
\usepackage{enumitem}
% For calculations (such as textwidth).
\usepackage{calc}
% For managing math fonts (list symbols).
\usepackage[math-style=TeX,vargreek-shape=unicode]{unicode-math}
% For conditional statements.
\usepackage{xifthen}




% ============================================================================
% CONFIGURE FONTS
% ============================================================================
% Configure directory location for fonts. [Default: 'fonts/']
\newcommand*{\fontdir}[1][fonts/]{\def\@fontdir{#1}}
\fontdir
\defaultfontfeatures{Ligatures=TeX}

% Set font for header.
\newfontfamily\headerfont[
  Path=\@fontdir,
  UprightFont=*-Regular,
  ItalicFont=*-Italic,
  BoldFont=*-SemiBold,
]{Raleway}

% Set main font.
\setmainfont[
  Path=\@fontdir,
  UprightFont=*-Regular,
  ItalicFont=*-Italic,
  BoldFont=*-SemiBold,
]{SourceSansPro}




% ============================================================================
% FORMATTING DEFINITIONS (for custom sections and entries)
% ============================================================================
% Define environment for Description items.
\newenvironment{Description}{%
  \vspace{\gapDescriptionBefore}
  \begin{justify}
  \begin{itemize}[label=\textbullet,leftmargin=2ex,nosep]
}{%
  \end{itemize}
  \vspace{\gapDescriptionAfter}
  \end{justify}
}

\newenvironment{JobDescription}{%
  \vspace{\gapJobDescriptionBefore}
  \begin{justify}
  \begin{itemize}[label=\textbullet,leftmargin=2ex,nosep]
}{%
  \end{itemize}
  \vspace{\gapJobDescriptionAfter}
  \end{justify}
}


% ----------------------------------------------------------------------------
% Define environment for Education section.
\newenvironment{EducationSection}{%
  \vspace{\gapAfterSectionTitle}
  \begin{center}
}{%
  \end{center}
}

% Define Education entry formatting.
% Usage: \Degree{<1-date>}{<2-location>}{<3-school>}{<4-major>}{<5-description>}
\newcommand*{\Degree}[5]{%
  \vspace{\gapBetweenEntry}
  \setlength\tabcolsep{0pt}
  \setlength{\extrarowheight}{0pt}
  \begin{tabular*}{\textwidth}{@{\extracolsep{\fill}} L{\textwidth-4.5cm} R{4.5cm}}
    \styleSchoolName{#3} & \styleLocation{#2} \\
      \styleDegreeTitle{#4} & \styleDate{#1} \\
      \multicolumn{2}{L{0.95\textwidth}}{\styleDescription{#5}}
  \end{tabular*}%
}


% ----------------------------------------------------------------------------
% Define environment for Experience section.
\newenvironment{ExperienceSection}{%
  \vspace{\gapAfterSectionTitle}
}

% Define Experience entry formatting.
% Usage: \Job{<1-date>}{<2-location>}{<3-company>}{4-subcompany}{<5-position>}{<6-description>}
\newcommand*{\Job}[6]{%
  \vspace{\gapBetweenEntry}
  \begin{tabular}{l|p{0.8\textwidth}}
    \ifthenelse{\isempty{#4}}
      {\styleDate{#1} & \styleJobTitle{#5}\formatSeparator
            \styleCompanyName{#3} \\
        \styleLocation{#2} & \styleDescription{#6}}
      {\styleDate{#1} & \styleJobTitle{#5}\formatSeparator
            \styleCompanyName{#3} (\styleSubCompany{#4}) \\
        \styleLocation{#2} & \styleDescription{#6}}
  \end{tabular}%
}


% ----------------------------------------------------------------------------
% Define environment for Projects section.
\newenvironment{ProjectSection}{%
  \vspace{\gapAfterSectionTitle}
  \begin{center}
}{%
  \end{center}
}

% Define Project entry formatting.
% Usage: \Project{<title>}{<extra-info>}{<description>}
\newcommand*{\Project}[3]{%
  \vspace{\gapBetweenEntry}
  \setlength\tabcolsep{0pt}
  \setlength{\extrarowheight}{0pt}
  \begin{tabular*}{\textwidth}{@{\extracolsep{\fill}} L{\textwidth-4.5cm} R{4.5cm}}
    \ifthenelse{\isempty{#2}}
      {\styleProjectName{#1} \\
        \multicolumn{2}{L{0.98\textwidth}}{\styleDescription{#3}}}
      {\styleProjectName{#1} (\styleProjectExtra{#2}) \\
        \multicolumn{2}{L{0.95\textwidth}}{\styleDescription{#3}}}
  \end{tabular*}%
}


% ----------------------------------------------------------------------------
% Define environment for Honors & Awards section.
%\newenvironment{AwardSection}{%
%  \vspace{\gapAfterSectionTitle}
%  \vspace{\gapBetweenEntry}
%  \begin{center}
%    \setlength\tabcolsep{0pt}
%    \setlength{\extrarowheight}{0pt}
%    \begin{tabular*}{\textwidth}{@{\extracolsep{\fill}} C{1.5cm} L{\textwidth-4.0cm} R{2.5cm}}
%}{%
%    \end{tabular*}
%  \end{center}
%}
%
%% Define honor/award entry.
%% Usage: \Award{<rank>}{<title>}{<location>}{<date>}
%\newcommand*{\Award}[4]{%
%  {#4} & {#1}, {#2} & {#3} \\
%}


% ----------------------------------------------------------------------------
% Define environment for Skills section.
\newenvironment{SkillSection}{%
  \vspace{\gapAfterSectionTitle}
  \vspace{\gapBetweenEntry}
  \begin{center}
    \setlength{\tabcolsep}{1em}
    \setlength{\extrarowheight}{0pt}
    \begin{tabular*}{\textwidth}{@{\extracolsep{\fill}} r L{0.75\textwidth}}
}{%
    \end{tabular*}
  \end{center}
}


% Define Skill entry formatting (single line).
% Usage: \Skill{<category>}{<skill-set>}
\newcommand*{\Skill}[2]{%
  \styleSkillCategory{#1} & \styleSkillSet{#2} \\
}




% ============================================================================
% FORMATTING OPTIONS
% ============================================================================
% For non-numbered pages.
\pagestyle{empty}

% Define header spacing.
\newcommand{\gapHeaderName}{1mm}
\newcommand{\gapHeaderInfo}{1mm}
\newcommand{\gapIcon}{\space}

% Define generic section spacing.
\newcommand{\gapBetweenSection}{0mm}
\newcommand{\gapAfterSectionTitle}{-1mm}
\newcommand{\gapBetweenEntry}{1mm}
\newcommand{\gapDescriptionBefore}{-4mm}
\newcommand{\gapDescriptionAfter}{-4mm}

% Define custom section spacing.
\newcommand{\gapJobDescriptionBefore}{-2mm}
\newcommand{\gapJobDescriptionAfter}{-5mm}

% Miscellaneous formatting shortcuts.
\newcommand{\formatSeparator}{\enskip\textbar\enskip}

% Customize section titles.
\titleformat{\section}{\headerfont\Large\scshape\raggedright}{}{0em}{}[\titlerule]
\titlespacing{\section}{0pt}{3pt}{3pt}

% Use to align tabular elements.
\newcolumntype{L}[1]{>{\raggedright\let\newline\\\arraybackslash\hspace{0pt}}m{#1}}
\newcolumntype{C}[1]{>{\centering\let\newline\\\arraybackslash\hspace{0pt}}m{#1}}
\newcolumntype{R}[1]{>{\raggedleft\let\newline\\\arraybackslash\hspace{0pt}}m{#1}}




% ============================================================================
% STYLING DEFINITIONS
% ============================================================================
\newcommand*{\styleHeaderName}[1]{{\headerfont\Huge\bfseries #1}}
\newcommand*{\styleHeaderInfo}[1]{{\headerfont\normalsize #1}}

\newcommand*{\styleLocation}[1]{{\normalsize\itshape #1}}
\newcommand*{\styleDate}[1]{{\normalsize #1}}
\newcommand*{\styleDescription}[1]{{\small #1}}

% For Education entry elements.
\newcommand*{\styleSchoolName}[1]{{\large\bfseries #1}}
\newcommand*{\styleDegreeTitle}[1]{{\large\scshape #1}}

% For Experience entry elements.
\newcommand*{\styleCompanyName}[1]{{\large\scshape #1}}
\newcommand*{\styleJobTitle}[1]{{\large\bfseries #1}}
\newcommand*{\styleSubCompany}[1]{{\normalsize #1}}

% For Project entry elements.
\newcommand*{\styleProjectName}[1]{{\large\bfseries #1}}
\newcommand*{\styleProjectExtra}[1]{{\normalsize #1}}

% For Skill entry elements.
\newcommand*{\styleSkillCategory}[1]{{\normalsize\bfseries #1}}
\newcommand*{\styleSkillSet}[1]{{\normalsize #1}}




% ============================================================================
% MAIN DOCUMENT
% ============================================================================
\begin{document}

% ----------------------------------------------------------------------------
% HEADER
% ----------------------------------------------------------------------------
\begin{center}
  \styleHeaderName{Kenny Ng}%
  \\[\gapHeaderName]%
  \styleHeaderInfo{%
    % Phone Number
    \faMobile \gapIcon +1 (000) 000-0000%
    \formatSeparator%
    % E-mail Address
    \faEnvelopeO \gapIcon \href{mailto:name@college.edu}{name\textit{@}college.edu}%
    \formatSeparator%
    % Personal Website
    \faHome \gapIcon \href{http://www.kennyng.org}{kennyng.org}%
    \formatSeparator%
    % Github
    \faGithub \gapIcon \href{https://github.com/kennyng}{github.com/kennyng}%
  } \vspace{\gapHeaderInfo}%
\end{center}




% ----------------------------------------------------------------------------
% SECTION: Education
% ----------------------------------------------------------------------------
\section{Education}
\begin{EducationSection}
  \Degree
    {Month YEAR} % Date(s)
    {Stanford, CA} % Location
    {Stanford University} % Institution
    {B.S. in Mathematical and Computational Science} % Degree
    {
      \begin{Description} % Extra Information
        \item {Relevant Coursework:}
            \subitem {Each subitem is a non-bulleted list of courses.}
            \subitem {List only the most important and most relevant courses.}
      \end{Description}
    }
\end{EducationSection}




% ----------------------------------------------------------------------------
% SECTION: Experience
% ----------------------------------------------------------------------------
\section{Experience}
\begin{ExperienceSection}
  \Job
    {01/2017 -- 02/2017} % Date(s)
    {San Francisco, CA} % Location
    {MyCompany} % Company
    {CompanyInfo} % Company Affiliations/Alias
    {Software Engineer} % Position
    {
      \begin{JobDescription} % Description of Tasks/Responsibilities
        \item {Each item is bulleted with no indent.}
            \subitem {Subitem is indented with no bullet.}
        \item {Describe what you did using active verbs and quantify results.}
        \item {Optionally: list some of the tools and technologies used.}
      \end{JobDescription}
    }

  \Job
    {06/2016 -- 08/2016} % Date(s)
    {San Francisco, CA} % Location
    {MyCompany} % Company
    {CompanyInfo} % Company Affiliations/Alias
    {Software Engineer} % Position
    {
      \begin{JobDescription} % Description of Tasks/Responsibilities
        \item {Each item is bulleted with no indent.}
            \subitem {Subitem is indented with no bullet.}
        \item {Describe what you did using active verbs and quantify results.}
        \item {Optionally: list some of the tools and technologies used.}
      \end{JobDescription}
    }
\end{ExperienceSection}




% ----------------------------------------------------------------------------
% SECTION: Projects
% ----------------------------------------------------------------------------
\section{Projects}
\begin{ProjectSection}
  \Project
    {My Awesome Project} % Project Title
    {extra info} % Extra Info
    {
      \begin{Description} % Project Description
        \item {Describe the project: objective, goals, etc.}
        \item {Explain what you did; describe achievements or results (quantify).}
      \end{Description}
    }

  \Project
    {\href{https://github.com/kennyng}{My Awesome Project Link}} % Project Title
    {extra info} % Extra Info
    {
      \begin{Description} % Project Description
        \item {Describe the project: objective, goals, etc.}
        \item {Explain what you did; describe achievements or results (quantify).}
      \end{Description}
    }
\end{ProjectSection}




% ----------------------------------------------------------------------------
% SECTION: Skills
% ----------------------------------------------------------------------------
\section{Skills}
\begin{SkillSection}
    \Skill
        {Spoken Languages:} % Category
        {English, Cantonese Chinese, Mandarin Chinese, Spanish} % List of Skills

    \Skill
        {Programming Languages:} % Category
        {Python, C, JavaScript, R, Go, Java, SQL, MATLAB, C++, C\#, Haskell} % List of Skills

    \Skill
        {Technologies:} % Category
        {List of items related to the category} % List of Skills
\end{SkillSection}

\end{document}
